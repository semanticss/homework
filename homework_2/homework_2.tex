\documentclass[12pt]{article}

\usepackage{amsmath}
\usepackage[margin=1in]{geometry}
\usepackage{fancyhdr}
\usepackage{amsthm}
\usepackage{braket}
\usepackage{amssymb}
\usepackage{tikz}
\usetikzlibrary{shapes.geometric}
\usepackage{enumerate}
\usepackage[shortlabels]{enumitem}


\pagestyle{fancy}
\fancyhead[l]{Hudson Benites}
\fancyhead[c]{Homework \#2}
\fancyfoot[c]{\thepage}
\renewcommand{\headrulewidth}{0.2pt}
\setlength{\headheight}{15pt}
\fancyhead[r]{\today}

\begin{document}
NOTE: ALL ANSWERS TEMPORARY AND NOT CHECKED
\begin{enumerate}[start = 1, label = {\bfseries Question \arabic*:}, leftmargin=1in]
    \item (Section 2.3, \#2b) $B = \Set{-\pi^x | x \leq 0}$.
    \item (Section 2.3, \#5)
        \begin{itemize}
            \item[(a)] $S = \Set{x \in \mathbb{Z} | x \geq 5}$.
            \item[(b)] $S = \Set{2x + 1 | x \in \mathbb{Z}}$.
            \item[(c)] $S = \Set{x \in \mathbb{Q} | x > 0}$.
            \item[(d)] $S = \Set{x \in \mathbb{R} | 1 < x < 7}$.
            \item[(e)] $S = \Set{x \in \mathbb{R} | x^2 > 0}$.
        \end{itemize}

    \item (Section 5.3, \#2)
        \begin{proof}
        In order to prove equality, we show that $A\cup (B \cap C) \subseteq (A \cap B) \cup (A \cap C)$ and $(A \cap B) \cup (A \cap C) \subseteq A\cup (B \cap C)$.
        \begin{itemize}
            \item[($\rightarrow$)] Suppose $x$ is an arbitrary element in $A \cap (B \cup C)$. Then, $x\in A \land (x\in B \lor x\in C)$. By the distributive property of conjunctions and disjunctions, $(x\in A \land x\in B) \lor (x\in A\lor x\in C)$. By the definition
            of
        \end{itemize}
        \end{proof}

    \item (Should be question 6)
        \begin{itemize}
            \item[(a)] False. $11 \notin B$.
            \item[(b)] True. Every element in $A$ is also in $\mathbb{Z}^+$.
            \item[(c)] False. $A$ is not an integer, and therefore is not a subset of $\mathbb{Z}^+$. 
            \item[(d)] True.
        \end{itemize}

    \item (Should be question 7)
        \begin{itemize}
            \item[(a)] False. The empty set has no elements, and the set on the RHS has 1 element.
            \item[(b)] True. The empty set is a subset of every set.
            \item[(c)] True. The empty set appears in the set on the RHS, and therefore is an element.
        \end{itemize}

    \item (Should be question 8)
    \begin{proof}
    Suppose $A$ and $B$ are sets, and $A \subseteq B$. Let $X$ be an arbitrary element of $\mathcal{P}(A)$. Then, $X \subseteq A \subseteq B$, so $X\subseteq B$. Thus, $X\in \mathcal{P}(B)$. So, an aribtrary element of $\mathcal{P}(A)$ is in $\mathcal{P}(B)$, so $\mathcal{P}(A)\subseteq \mathcal{P}(B)$.  
        
    \end{proof}

    \item (Should be question 9)
        \begin{proof}
        In order to show that $\mathcal{P}(A \cap B) = \mathcal{P}(A) \cap \mathcal{P}(B)$, we show that they are both are a subset of each other.

        \begin{itemize}
            \item[$(\subseteq)$] Suppose $A$ and $B$ are sets, and $C\in \mathcal{P}(A \cap B)$. Then, $C\subseteq(A\cap B) \rightarrow C\subseteq A$ and $C\subseteq B$. $C\subseteq A \rightarrow C\in \mathcal{P}(A)$. Similarly,
            $C\subseteq B \rightarrow C\in \mathcal{P}(B)$. Thus, $C \in \mathcal{P}(A) \cap \mathcal{P}(B)$.
            
            \item[$(\supseteq)$] The argument is symmetric.
        \end{itemize}
        \end{proof}

    \item (Should be question 10)
        \begin{itemize}
            \item[] $X \times Y = \Set{(a,a), (a,b), (a,d), (c,a), (c,b), (c,d)}$.
            \item[] $Y \times X = \Set{(a,a), (b,a), (d,a), (a,c), (b,c), (d,c)}$.
            \item[] $X \times X = \Set{(a,a), (a,c), (c,a), (c,c)}$.
        \end{itemize}

    \item (Should be question 12)
        \begin{itemize}
            \item[] $\bigcap\limits_{k=1}^{\infty}A_k = \{1\}$
            \item[] $\bigcup\limits_{k=1}^{\infty}A_k = [1,2]$
            \item[] $\bigcap\limits_{k=1}^{\infty}B_k = \varnothing$
            \item[] $\bigcup\limits_{k=1}^{\infty}A_k = (1,2)$
        \end{itemize}

    \item (Should be question 13)
        \begin{proof} 
            Let $S$ be square with vertices $ABCD$ and side length $1$, and let there be 5 points, $p_1, p_2,\ldots, p_5$.  Construct 4 circles of radius $\frac{\sqrt{2}}{2}$, centered about $A, B, C, D$ respectively. We place one point from $p1,\dots, p_4$ into each shaded region. $p_5$ must go into 

    \end{proof}
\end{enumerate}
\end{document}