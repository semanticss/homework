\documentclass[12pt]{article}

\usepackage{amsmath}
\usepackage[margin=1in]{geometry}
\usepackage{fancyhdr}
\usepackage{amsthm}
\usepackage{braket}
\usepackage{amssymb}
\usepackage{tikz}
\usepackage{mathtools}
\usetikzlibrary{shapes.geometric}
\usepackage{enumerate}
\usepackage[shortlabels]{enumitem}


\pagestyle{fancy}
\fancyhead[l]{Hudson Benites}
\fancyhead[c]{Homework \#2}
\fancyfoot[c]{\thepage}
\renewcommand{\headrulewidth}{0.2pt}
\setlength{\headheight}{15pt}
\fancyhead[r]{\today}

\begin{document}
\begin{enumerate}[start = 1, label = {\bfseries Question \arabic*:}, leftmargin=1in, itemsep=1.5em]
    \item \begin{itemize}
        \item[(a)] $[18] = \Set{4n+2 | n\in \mathbb{Z}}$.
        \item[(b)] $[31] = \Set{4n+2 | n\in \mathbb{Z}}$.
        \item[(c)] There are 4 equivalence classes. 
    \end{itemize}

    \item \begin{proof}
        Need to prove that the relation is reflective, symmetric, and transitive. 

    \end{proof}

    \item \begin{proof}
        Suppose $a,b,c,d \in \mathbb{Z}$, and $a\equiv_n c$ and $b\equiv_n d$. Then,
        \begin{center}
            $a-c=nk, k\in \mathbb{Z}$\\ $b-d=nl, l\in \mathbb{Z}$.
        \end{center}
        Adding the equations, we get 
        \begin{center}
        $a+b-c-d=n(k+l)$.
        \end{center} Simplifying, we get \begin{center}
            $(a+b)-(c+d) = n(k+l)$.
        \end{center}
        Thus, by definition $a+b \equiv_n c+d$.
    \end{proof}

    \item \begin{itemize}
        \item[(a)] \begin{proof}
        The theorem given in the proposition is equivalent to claiming that \lq{If $ab$ is even, then $a$ is even or $b$ is even\rq}. We have proven this theorem to be true, so it is the case that if $ab \equiv_2 0$, then $a \equiv_2 0$ or $b \equiv_2 0$.
        \end{proof}
    \end{itemize}
        
\end{enumerate}
\end{document}