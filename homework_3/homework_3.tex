\documentclass[12pt]{article}

\usepackage{amsmath}
\usepackage[margin=1in]{geometry}
\usepackage{fancyhdr}
\usepackage{amsthm}
\usepackage{braket}
\usepackage{amssymb}
\usepackage{tikz}
\usetikzlibrary{shapes.geometric}
\usepackage{enumerate}
\usepackage[shortlabels]{enumitem}


\pagestyle{fancy}
\fancyhead[l]{Hudson Benites}
\fancyhead[c]{Homework \#2}
\fancyfoot[c]{\thepage}
\renewcommand{\headrulewidth}{0.2pt}
\setlength{\headheight}{15pt}
\fancyhead[r]{\today}

\begin{document}
\begin{enumerate}[start = 1, label = {\bfseries Question \arabic*:}, leftmargin=1in]
    \item (Section 6.4, \#2)
        \begin{itemize}
            \item[-] $g \circ h = g(h(x)) = g(3x+2)={(3x+2)}^2$.
            \item[-] $h\circ g = h(g(x)) = h(x^3)=3x^3+2$.
            \item[-] $g\circ h \neq h\circ g$, thus composition is not $\ldots$.
        \end{itemize}

    \item
        \mbox{}
        \begin{itemize}
            \item[(a)] The image of $f$ is all the odd integers. The image of $f(X)$ where $X=\Set{x\in \mathbb{Z} | x \text{ is even}}$ is every other odd integer.
            \item[(b)] The image of $f$ is every positive real number. The image of $f(X)$ where $X = (0,1]$ is $(\infty, 1]$. % chktex 9
            \item[(c)] The image of $f$ is $[-1,1]$. The image of $f(X)$ where $X = [0,\pi)$ is also $[-1,1]$. % chktex 9
        \end{itemize}

    \item
        \mbox{}
        \begin{enumerate}
            \item[(a)]
                \begin{proof}
                Suppose $A$ and $B$ are sets with subsets $U,V \subseteq A$, and $f : A \to B$ is a function. Let $b$ be an aribtrary element of $f(U \cap V)$. This means that there 
                exists an $x \in U \cap V$ such that $f(x) = b$. Therefore, $\exists x \in U$ such that $f(x) = b$, and $\exists x\in V$ such that $f(x)=b$. Therefore, $b\in f(U) \land b\in f(V) \rightarrow b\in f(U) \cap f(V)$. So, $f(U \cap V ) \subseteq f(U) \cap f(V)$.
                \end{proof}
        \end{enumerate}

    \item
        \mbox{}
        \begin{enumerate}
            \item[(a)] $f^{-1}(Y) = [0,\infty)$. % chktex 9
            \item[(b)] I think this is like all $[n\pi, (n+1)\pi]$ only for even $n$. 
            \item[(c)] I think this is $\Set{1,2,10,12,14,16,18}$. 
        \end{enumerate}
\end{enumerate}
\end{document}